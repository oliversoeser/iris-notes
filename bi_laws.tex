\documentclass[12pt]{article}
\usepackage[a4paper]{geometry}

\usepackage[english]{babel}
\usepackage{amsthm}
\usepackage{amsmath}
\usepackage{amssymb}
\usepackage{amsfonts}

\newcommand{\ent}{\vdash}
\renewcommand{\phi}{\varphi}
\newcommand{\pure}[1]{\ulcorner #1 \urcorner}
\newcommand{\imp}{\Rightarrow}
\renewcommand{\and}{\land}

\title{Laws of the Logic of Bunched Implications in Iris}
\author{Oliver Soeser}
\date{\today}

\begin{document}

\maketitle

First of all, the entailment relation $\ent$ is a preorder, giving us the reflexivity and transitivity laws
\begin{equation*}
  P \ent P, \qquad \frac{P \ent Q \quad Q \ent R}{P\ent R}.\\
\end{equation*}
Pure propositions have the straightforward introduction and elimination rules
\begin{equation*}
  \frac{\phi}{P \ent \pure\phi},\qquad \frac{\phi \imp  (True \ent P)}{\pure\phi \ent P}.
\end{equation*}
The usual rules apply for standard conjunction, disjunction, and implication:

\end{document}